\chapter{Experiments}

% Short introduction such as: this chapter describes how the tests should be
% conducted, explain thorougly the scheduler script, gathered data etc.
\section{Proposed workflow and methodology}

In order to obtain the details of how system-level physical measurement
estimates energy consumption by a component (such as a CPU or a GPU)
during an application execution, several steps must be taken:

\begin{enumerate}
    \item Exclusive reservation of the entire computational node.
    \item Observation of the disk consumption and network usage before and
    during tests.
    \item Monitoring of the CPUs and GPUs utilization before and during tests.
    \item Running the benchmark kernels on an abstract processor only.
    Abstract processor comprises of the multicore CPU processor consisting
    of a certain number of physical cores and DRAM\@.
    \item Gathering of the power measurements.
    \item Verification of the accuracy and reliability of the software
    measurements tools, based on ground truth results.
\end{enumerate}

One of the notable mentions that could be done in order to reduce the amount
of uncertain power draw measurements done by the background components is
setting the fans to full speed. This solution have a potential of reducing
power draw fluctuations, especially during higher workloads, i\. e\. when
running benchmarks kernels utilizing maximum amount of GPUs or running Hybrid
configuration, where power draw of the entire nodes are very high. This could
not be implemented, however, as the administrator of the department's servers
stated, that interference in servers fans could be crucial for the nodes
stability.

\textbf{For Intel RAPL / NVIDIA Management Library:}
\begin{enumerate}
    \item Obtain base power of idle CPUs / GPUs.
    \item Obtain execution time of benchmark application.
    \item Obtain total energy consumption of the CPUs / GPUs during tests.
    \item Calculate dynamic energy consumption by subtracting base energy from
    total energy used during run.
\end{enumerate}

\textbf{For Yokotool:}
\begin{enumerate}
    \item Obtain base power of idle CPUs.
    \item Obtain execution time of benchmark application.
    \item Obtain total energy consumption of the CPUs during tests.
    \item Calculate dynamic energy consumption by subtracting base power from
    total energy used during run
\end{enumerate}

\newpage

In addition to the main experiments workflow, another methodology must be
adapted $-$ the data collection methodology. In order for the results to be
properly comparable, several point have to be met:
\begin{enumerate}
    \item Tests environment must be identical in every case, to eliminate
    discrepancy of the results.
    \item The results of the power draw reading must be properly compared for
    the device only measurements (Intel RAPL / NVML) and the measurements of
    the entire node (Yokotool).
    \item Experiments should be conducted on different nodes that utilizes
    different hardware, in order to state repeatability of tests. [TO BE REDACTED]
    \item Experiments should be conducted, using different benchmark kernels
    or application, to remove the possibility of bias of the results, due to
    poor diversification of test cases. [TO BE REDACTED]
    \item Test runs must be repeated many times.
\end{enumerate}

\section{Working environment $-$ servers details}

[PLACEHOLDER]
% Describe more thoroughly the nodes: CPUs, GPUs, RAM, PSUs, PM connection etc

\section{Main tests}

\subsection{Overview on the scheduler script}

Things to explain:
1. Dictionaries with configs, that work for both choosing the right config and
provides path to save measurements
2. Section that run CPUs benchmarks, GPUs benchmarks, perf, nvml, yoko
3. Functions that check if the benchmarks are still running
4. Function that cleans-up every process after tests


\subsection{Main automation function $-$ scheduler()}

% NOTE TO SELF: Make sure this chart belongs to this section

\begin{figure}[hbtp]
    \centering
    \includegraphics{general_flowchart}
    \caption{General Flowchart}~\label{fig:general_flowchart}
\end{figure}


\subsection{Threads pinning and kernels execution $-$ cpu\_benchmark()}

\begin{figure}[hbtp]
    \centering
    \includegraphics{processes_flowchart.jpeg}
    \caption{Processes Flowchart}~\label{fig:processes_flowchart}
\end{figure}

% ===================

\subsection{GPUs and threads management $-$ gpu\_benchmark\@()}

\subsection{Measurements with Yokotool software $-$ yoko\@()}

\subsection{Measurements with Linux Perf software $-$ perf\@()}

\subsection{Measurements with NVML handling function $-$ nvml\@()}

\subsection{Cleanup of measurements daemons}

\subsection{Termination of benchmarks in Hybrid configuration}

% psutil

[PLACEHOLDER]
% Put the charts and plots of received measurements

\newpage

\section{Analysis of the results and discussion}

\begin{table}[hbt!]
    \rowcolors{1}{Lavender!80!gray}{white}
    \centering
    \caption{sanna.kask, CPUs, OMP-CPP, bt.C, 1 CPU [POWER DRAW ONLY!!!]}\label{tbl:sanna.kask_CPUs_OMP-CPP_bt.C}
    \setlength{\tabcolsep}{5mm}
    \begin{tblr}{
        % {|l|c|c|c|c|},
        % hlines,
        vlines,
        row{1}={font=\bfseries,halign=c,bg=lightgray!30},
        row{7-8} = {bg = lightgray!20}
        }
    \hline
        & \SetCell[c=4]{c} 1 CPU  \\
    \hline
        Results from 10 runs                                    & 1 Thread  & 5 Threads & 10 Threads    & 20 Threads \\
    \hline
        {Avg. Exec\@. time [s]}                                 & 1054.795  & 216.315   & 114.315       & 101.372 \\
    \hline
        {Std\@. dev\@. of time [-]}                             & 0.966     & 0.243     & 0.121         & 0.158 \\
    \hline
        {(Yokogawa) \\ Avg\@. power draw [W]}                   & 379.962   & 402.881   & 432.171       & 445.752 \\
    \hline
        {(Yokogawa) \\ Std\@. dev\@. of avg\@. power draw [-]}  & 0.605     & 0.225     & 0.382         & 1.007 \\
    \hline
        {(CPU\@: 0) \\ Avg\@. power draw [W]}                   & 33.717    & 51.274    & 70.433        & 77.958 \\
    \hline
        {(CPU\@: 0) \\ Std\@. dev\@. of avg\@. power draw [-]}  & 0.102     & 0.13      & 0.085         & 0.089 \\
    \hline
        {(CPU\@: 1) \\ Avg\@. power draw [W]}                   & 28.98     & 28.887    & 28.871        & 28.874 \\
    \hline
        {(CPU\@: 1) \\ Std\@. dev\@. of avg\@. power draw [-]}  & 0.125     & 0.062     & 0.067         & 0.055 \\
    \hline
        {(GPU\@: 0) \\ Avg\@. power draw [W]}                   & 21.755    & 21.673    & 21.604        & 21.634 \\
    \hline
        {(GPU\@: 0) \\ Std\@. dev\@. of avg\@. power draw [-]}  & 0.343     & 0.061     & 0.2           & 0.05 \\
    \hline
        {(GPU\@: 1) \\ Avg\@. power draw [W]}                   & 25.522    & 25.534    & 25.516        & 25.524 \\
    \hline
        {(GPU\@: 1) \\ Std\@. dev\@. of avg\@. power draw [-]}  & 0.023     & 0.032     & 0.052         & 0.029 \\
    \hline
        {(GPU\@: 2) \\ Avg\@. power draw [W]}                   & \\
    \hline
        {(GPU\@: 2) \\ Std\@. dev\@. of avg\@. power draw [-]}  & \\
    \hline
        {(GPU\@: 3) \\ Avg\@. power draw [W]}                   & \\
    \hline
        {(GPU\@: 3) \\ Std\@. dev\@. of avg\@. power draw [-]}  & \\
    \hline
        {(GPU\@: 4) \\ Avg\@. power draw [W]}                   & \\
    \hline
        {(GPU\@: 4) \\ Std\@. dev\@. of avg\@. power draw [-]}  & \\
    \hline
        {(GPU\@: 5) \\ Avg\@. power draw [W]}                   & \\
    \hline
        {(GPU\@: 5) \\ Std\@. dev\@. of avg\@. power draw [-]}  & \\
    \hline
        {(GPU\@: 6) \\ Avg\@. power draw [W]}                   & \\
    \hline
        {(GPU\@: 6) \\ Std\@. dev\@. of avg\@. power draw [-]}  & \\
    \hline
        {(GPU\@: 7) \\ Avg\@. power draw [W]}                   & \\
    \hline
        {(GPU\@: 7) \\ Std\@. dev\@. of avg\@. power draw [-]}  & \\
    \hline
    \end{tblr}
\end{table}


% \begin{table}[hbt!]
%     % \centering
%     % \small
%     \caption{sanna.kask, CPUs, OMP-CPP, bt.C [POWER DRAW ONLY!!!]}\label{tbl:sanna.kask_CPUs_OMP-CPP_bt.C}
%     \begin{tblr}{|c|c|c|c|c|c|c|c|c|}
%     \hline
%         & \SetCell[c=4]{c} 1 CPU & & & & \SetCell[c=4]{c} 2 CPUs \\
%     \hline
%         (10 runs) & 1 Thread & 5 Threads & 10 Threads & 20 Threads & 2 Threads & 10 Threads & 20 Threads & 40 Threads \\
%     \hline
%         {Avg. \\ Exec\@. time [s]} & \\
%     \hline
%         {Std\@. dev\@. \\ of time [-]} & \\
%     \hline
%         {(Yokogawa) Avg\@. \\ power draw [W]} & \\
%     \hline
%         {(Yokogawa) Std\@. \\ dev\@. of avg\@. \\ power draw [-]} & \\
%     \hline
%         {(CPU:0) Avg\@. \\ power draw [W]} & \\
%     \hline
%         {(CPU:0) Std\@. dev\@. \\ of avg\@. \\ power draw [-]} & \\
%     \hline
%         {(CPU:1) Avg\@. \\ power draw [W]} & \\
%     \hline
%         {(CPU:1) Std\@. dev\@. \\ of avg\@. \\ power draw  [-]} & \\
%     \hline
%     \end{tblr}
% \end{table}

[PLACEHOLDER]
% Put the analysis of the received data, according to Dr. Czarnul