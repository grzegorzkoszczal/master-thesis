\chapter*{Introduction}
\addcontentsline{toc}{chapter}{Introduction}

The Information and Communication Technology sector is responsible
for a significant share of global electricity use. In 2020,
the data centers, communication networks and user devices
accounted for an estimated 4$-$6\% of global electricity
use~\cite{Energy_Consumption_of_ICT}. This value has grown
exponentially over the last years, mainly due to technological
advances such as cloud computing and the rapid growth of the use of
Internet services~\cite{Data_Centre_Energy_Consumption}. Moreover,
it is predicted that ICT energy use is likely to increase
until 2030 and may reach approximately 13\% of global electricity
use. According to the increased energy demand, vast efforts has
been put in order to improve the energy efficiency across the
ICT sector with a success. As a result the overall energy use
remains mostly flat according  to some estimates.

The goal of energy efficiency increase is reached using various
methods. Some of them involve the kernels optimizations and
using the energy methods of
computing~\cite{Software_Methods_of_Energy_Efficiency}.
Such methods focus mainly on load imbalance, mixed precision
in floating-point operations. Other methods of increasing
computational efficiency are related to
power-capping~\cite{Parallel_Apps_Power_Capping_Czarnul}.
Such an approach assumes setting a certain power limit level on
CPU or GPU in order to achieve a power/performance trade-off.
Appropriate limitations of power draw results in slightly
longer execution times and significant savings, which render
this method a viable option.

In order to make such implementation meaningful, we must make
sure that the tools used to validate them are also as precise
as possible. For the measurement of the power draw of a CPU,
Intel provides its own interface, called Running Average Power
Limit (RAPL)~\cite{RAPL_Reporting}. On the GPU side, NVIDIA
provides the NVIDIA Management Library (NVML)~\cite{NVML}.
Unfortunately, concerns rise on the precision of such softwares,
mainly because their providers don't share any information about
estimated error of power draw readings, leaving researchers
questioning their practical use. In order to verify the
precision of software measurement tools, the external
measurement tool is used $-$ the Yokogawa
WT310E~\cite{Yokogawa_Producent_Page}
~\cite{Yokogawa_Meter_Series_Specifications}.
Its high precision, backed up by certificates, makes it an
excellent tool to benchmark the precision of Intel RAPL and NVML\@.
