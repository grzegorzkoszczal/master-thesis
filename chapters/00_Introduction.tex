\chapter{Introduction}
% \addcontentsline{toc}{chapter}{Introduction}

The Information and Communication Technology sector is responsible
for a significant share of global electricity use. In 2020,
the data centers, communication networks and user devices
accounted for an estimated 4$-$6\% of global electricity
use~\cite{Energy_Consumption_of_ICT}. This value has grown
exponentially over the last years, mainly due to technological
advances such as cloud computing and the rapid growth of the use of
Internet services~\cite{Data_Centre_Energy_Consumption}. Moreover,
it is predicted that ICT energy use is likely to increase
until 2030 and may reach approximately 13\% of global electricity
use. According to the increased energy demand, vast efforts has
been put in order to improve the energy efficiency across the
ICT sector with a success. As a result the overall energy use
remains mostly flat according  to some estimates.

The goal of energy efficiency increase is reached using various
methods. Some of them involve the kernels optimizations and
using the energy methods of
computing~\cite{Software_Methods_of_Energy_Efficiency}.
Such methods focus mainly on load imbalance, mixed precision
in floating-point operations. Other methods of increasing
computational efficiency are related to
power-capping~\cite{Parallel_Apps_Power_Capping_Czarnul}.
Such an approach assumes setting a certain power limit level on
CPU or GPU in order to achieve a power/performance trade-off.
Appropriate limitations of power draw results in slightly
longer execution times and significant savings, which render
this method a viable option.

In order to make such implementation meaningful, we must make
sure that the tools used to validate them are also as precise
as possible. For the measurement of the power draw of a CPU,
Intel provides its own interface, called Running Average Power
Limit (RAPL)~\cite{RAPL_Reporting}. On the GPU side, NVIDIA
provides the NVIDIA Management Library (NVML)~\cite{NVML}.
Unfortunately, concerns rise on the precision of such softwares,
mainly because their providers don't share any information about
estimated error of power draw readings, leaving researchers
questioning their practical use. In order to verify the
precision of software measurement tools, the external
measurement tool is used $-$ the Yokogawa
WT310E~\cite{Yokogawa_Producent_Page}
~\cite{Yokogawa_Meter_Series_Specifications}.
Its high precision, backed up by certificates, makes it an
excellent tool to benchmark the precision of Intel RAPL and NVML\@.

\newpage

% \section{Research goal}

\section{Purpose and research question}

The goal of the project is to verify the accuracy and
reliability of the CPU and GPU power draw measurement
tools during the computational-straining benchmark applications.

\section{Scope and limitation}

The scope of the project is to verify whether the software
power measurement tools are precise, based upon the results
of the certified external measurement tool and to specify
their error range in case of inaccuracy.

In case of benchmark applications, a hypothesis worth
considering is whether the change of computational data
impacts the power draw measurements between hardware and
software tools. There are
experiments that proved that increasing
the data used in the benchmarks influences the increasingly
different results from the tools. Another hypothesis, however,
makes the claim that the application of power capping does
not impact the results. In order to make a conclusion, all
those configurations should be investigated independently.
Another aspect worth investigating is the use of various
power supply units, both the server and consumer grade. Those
tests would give us insight, whether they differ in total
power draw or is it more likely associated with the certificates
they have.

The project is limited to testing CPUs and software released
by Intel Corporation and to testing the GPUs and software
released by NVIDIA Corporation.

\section{Project requirements}

The project requires a reliable testbed in order to run the
computations, verified benchmark applications and certified
external measurement tool. Moreover, the computational station
must be exclusively reserved for the time of research in order
to prevent other user's applications from interfering in the
tests results, as well as the tests itself should be repeated
several times in order to maintain credibility.

The workstations on which the experiments will be conducted are
the two different university computational nodes. One of them
consists of two server grade CPUs and eight GPUs, while the second
one is equipped with two server grade CPUs and four GPUs. Both nodes
are adapted to use a custom power strip that supports the use
of external measurement tool.

The tests benchmark applications are considered suitable for the
purpose of the project, when they are able to strain the
workstation's hardware, using their entire computational resources.
The chosen applications for this task are mainly NAS Parallel
Benchmarks~\cite{NPB}~\cite{NASA_Advanced_Supercomputing} $-$ a set
of benchmarks designed to evaluate the performance of parallel
computing systems. The NPB suite contains a variety of
benchmarks, including linear algebra, FFT, stencil computations
and others. The benchmarks are intended to be representative of
some important real-world application problems and can be used
to assess the performance of various systems under different
conditions.

The external measurement tool is Yokogawa WT310E. It is a precise
and reliable equipment that will serve as the ground truth in
verification of reading from the software tools.

