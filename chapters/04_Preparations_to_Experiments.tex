\chapter{Preparations to experiments}

\section{Overview on all configurations}

The measurements will be conducted on many different setups, therefore it is
important to explain throughly the entire workflow.

\begin{itemize}
    \item Server $-$ The tests will be conducted on two computational
    nodes, that slightly differs from each other based on hardware equipped.
        \begin{itemize}
            \item sanna.kask
            \item vinnana.kask
        \end{itemize}
    \item Device $-$ Next, benchmarks will be organized based on devices,
    that will be utilized during tests.
        \begin{itemize}
            \item CPU\@(s)
            \item GPU\@(s)
            \item Hybrid (CPUs + GPUs)
        \end{itemize}
    \item Implementation $-$ Depending on current server, as well as device
    chosen, the implementation technology is specified.
        \begin{itemize}
            \item OMP-CPP (sanna.kask, CPU\@(s))
            \item OMP-CUDA (sanna.kask, GPU\@(s))
            \item OMP-CPP + OMP-CUDA (sanna.kask, CPU\@(s) + GPU\@(s))
            \item MPI-Fortran (vinnana.kask, CPU\@(s))
            \item Horovod-Python (vinnana.kask, GPU\@(s))
            \item MPI-Fortran + Horovod-Python (vinnana.kask, CPU\@(s) + GPU\@(s))
        \end{itemize}
    \item Benchmark and class size $-$ Picking the correct benchmark by the
    automation scheduler is based on implementation technology currently
    chosen as well device and server.
        \begin{itemize}
            \item OMP-CPP
            \begin{itemize}
                \item bt.C
                \item is.D
                \item lu.D
            \end{itemize}
        \end{itemize}
        \begin{itemize}
            \item OMP-CUDA
            \begin{itemize}
                \item lu.D
                \item sp.D
                \item ep.D
            \end{itemize}
        \end{itemize}
        \begin{itemize}
            \item Hybrid (OMP-CPP + OMP-CUDA)
            \begin{itemize}
                \item bt.C+lu.D
                \item is.D+sp.D
                \item lu.D+ep.D
            \end{itemize}
        \end{itemize}

        \begin{itemize}
            \item MPI-Fortran
            \begin{itemize}
                \item ep.D.x
                \item is.D.x
                \item lu.C.x
            \end{itemize}
        \end{itemize}
        \begin{itemize}
            \item Horovod-Python
            \begin{itemize}
                \item XCeption
            \end{itemize}
        \end{itemize}
        \begin{itemize}
            \item Hybrid (MPI-Fortran + Horovod-Python)
            \begin{itemize}
                \item ep.D.x+XCeption
                \item is.D.x+XCeption
                \item lu.C.x+XCeption
            \end{itemize}
        \end{itemize}
    \item Configuration $-$ lastly, the configuration of number of logical
    processors on CPU\@(s) and/or number of GPU\@(s) used in test are based on
    chosen implementation.
        \begin{itemize}
            \item for OMP-CPP implementation:
            \item 1 CPU 1 Thread
            \item 1 CPU 5 Threads
            \item 1 CPU 10 Threads
            \item 1 CPU 20 Threads
            \item 2 CPUs 2 Threads
            \item 2 CPUs 10 Threads
            \item 2 CPUs 20 Threads
            \item 2 CPUs 40 Threads
        \end{itemize}
        \begin{itemize}
            \item for OMP-CUDA implementation:
            \item 1 GPU 1 Thread
            \item 2 GPUs 2 Threads
            \item 4 GPUs 4 Threads
            \item 8 GPUs 8 Threads
        \end{itemize}
        \begin{itemize}
            \item for OMP-CPP+OMP-CUDA implementation:
            \item 1 CPU 4 Threads \& 1 GPU 1 Thread
            \item 1 CPU 8 Threads \& 2 GPUs 2 Threads
            \item 2 CPUs 16 Threads \& 4 GPUs 4 Threads
            \item 2 CPUs 32 Threads \& 8 GPUs 8 Threads
        \end{itemize}
        \begin{itemize}
            \item for MPI-Fortran implementation:
            \item Processes: 4
            \item Processes: 8
            \item Processes: 16
            \item Processes: 32
        \end{itemize}
        \begin{itemize}
            \item for Horovod-Python implementation:
            \item 1 GPU
            \item 2 GPUs
            \item 4 GPUs
        \end{itemize}
        \begin{itemize}
            \item for MPI-Fortran+Horovod-Python implementation:
            \item Processes: 8 \& 1 GPU
            \item Processes: 16 \& 2 GPUs
            \item Processes: 32 \& 4 GPUs
        \end{itemize}
\end{itemize}

\section{Choosing the correct configurations}
% Write stuff from the seminar
% Maybe cite some othe works, where benchmarks were short and tell why it was bas


\section{Preliminary tests}

\begin{table}[!ht]
    \centering
    \small
    \caption{Execution times of OMP-CPP benchmarks}\label{tbl:table-label}
    \begin{tblr}{%
        hlines,%
        vlines,%
        row{1}={font=\bfseries},%
        column{1}={halign=c},%
    }%
        Benchmark & Class size & Execution time [s] & Benchmark & Class size & Execution time [s] \\
        IS & Class B & 0.35 & FT & Class B & 2.92 \\
        IS & Class C & 1.25 & FT & Class C & 16.14 \\
        IS & Class D & 37.53 & FT & Class D & 371 \\

        EP & Class B & 3.06 & SP & Class B & 14.48 \\
        EP & Class C & 11.96 & SP & Class C & (freezed) \\
        EP & Class D & 177 & SP & Class D & (core dumped) \\

        CG & Class B & 7.77 & BT & Class B & 15.82 \\
        CG & Class C & 27.2 & BT & Class C & 55.182 \\
        CG & Class D & (freezed) & BT & Class D & 1185 \\

        MG & Class B & 2.32 & LU & Class B & 9.88 \\
        MG & Class C & 17.459 & LU & Class C & 34.88 \\
        MG & Class D & 185 & LU & Class D & 1111 \\
    \end{tblr}
\end{table}


\begin{table}[!ht]
    \centering
    \small
    \caption{Execution times of OMP-CUDA benchmarks}\label{tbl:OMP-CUDA}
    \begin{tblr}{%
        hlines,%
        vlines,%
        row{1}={font=\bfseries},%
        column{1,2,4,5}={halign=c},%
    }%
        Benchmark & Class size & Execution time [s] & Benchmark & Class size & Execution time [s] \\
        IS & C & (too short) & FT & C & 6.63 \\
        IS & D & 15.33 & FT & D & (unsuccesful) \\
        IS & E & (not defined) & FT & E & (unsuccesful) \\

        EP & C & (too short) & CG & C & (too short) \\
        EP & D & 27.16 & CG & D & (freezed) \\
        EP & E & 442.79 & CG & E & (freezed) \\

        MG & C & (too short) & LU & C & 16.89 \\
        MG & D & (unsuccesful) & LU & D & 300.74 \\
        MG & E & (unsuccesful) & LU & E & (unsuccesful) \\

        SP & C & 10.39 & BT & C & 12.19 \\
        SP & D & 220.86 & BT & D & (unsuccesful) \\
        SP & E & (unsuccesful) & BT & E & (unsuccesful) \\
    \end{tblr}
\end{table}

\begin{table}[!ht]
    \centering
    \small
    \caption{Execution times of MPI-Fortran benchmarks}\label{tbl:MPI-Fortran}
    \begin{tblr}{%
        hlines,%
        vlines,%
        row{1}={font=\bfseries},%
        column{1,2,4,5}={halign=c},%
    }%
        Benchmark & Class size & Execution time [s] & Benchmark & Class size & Execution time [s] \\
        IS & B & 0.18 & FT & B & 3.11 \\
        IS & C & 0.8 & FT & C & 13.83 \\
        IS & D & 17.02 & FT & D & 360.15 \\

        EP & B & 1.11 & CG & B & 3.33 \\
        EP & C & 4.39 & CG & C & 8.57 \\
        EP & D & 71.16 & CG & D & 761.02 \\

        MG & B & 0.48 & LU & B & 10.97 \\
        MG & C & 4.5 & LU & C & 41.81 \\
        MG & D & 88.34 & LU & D & 770.52 \\
    \end{tblr}
\end{table}


\begin{table}[!ht]
    \centering
    \small
    \caption{Execution times of Horovod-Python benchmarks}\label{tbl:Horovod-Python}
    \begin{tblr}{%
        hlines,%
        vlines,%
        row{1}={font=\bfseries},%
        column{1,2,4,5}={halign=c},%
    }%
        Number of GPUs & Number of iteration & Execution time [s] \\
        1 & 1 & time \\
        1 & 3 & time \\
        1 & 5 & time \\

        2 & 1 & time \\
        2 & 3 & time \\
        2 & 5 & time \\

        4 & 1 & time \\
        4 & 3 & time \\
        4 & 5 & time \\
    \end{tblr}
\end{table}




ep.D = 27.16 s          avg 155~160 [W]
sp.D = 220.86 s         avg 200~210 [W]
lu.D = 300.74 s         fluctuates between 200~230 [W]


check the avg power draws of horovod-python benchmark

OMP-CPP was run on 40 Threads
MPI-Fortran was run on 32 Threads, due to code structure

verfiy if cpp, cuda, F exec times has been correctly ported


1 GPU, 1 iteration - 124.2s acc: 88.99\%, gpu: util 90\% avg pwr draw: ~ 185 W
1 GPU, 3 iteration - 326.7s acc: 91.30\%
1 GPU, 5 iteration - 528.6s acc: 91.59\%

2 GPU, 1 iteration - 72.2s, acc: 82.9\%
2 GPU, 3 iteration - 176.2s, acc: 90.72\%
2 GPU, 5 iteration - 280.9s, 92.46\%     180 W

4 GPU, 1 iteration - 48.5s, acc: 76.81\%
4 GPU, 3 iteration - 101.1s, acc: 91.59\%
4 GPU, 5 iteration - 155.3s, acc: 93.91\% 180 W


