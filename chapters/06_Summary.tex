\chapter{Summary and future work}

\section{Summary}

[PLACEHOLDER]

\section{Future work}

In the terms of future work, this master thesis gives a strong insights for
further expansion of the experiments.

One of the ideas is to perform tests
on different node with more diverse components, such as different CPUs and
GPUs, than those tested, to check if the results can be repeated and compared
with already performed experiments. Other idea is to perform tests on both 
\emph{sanna.kask} and \emph{vinnana.kask} nodes simultaneously, using the
MPI and Horovod based benchmarks, due to their inter-node communication.
This approach would simulate running benchmarks on a HPC cluster. For
the configurations of this setup, it is a sound strategy to run the tests
utilising significant amount of the resources, such as tests on half of the
devices and then, utilising all of them on both servers. This will avoid
the repetition of already performed experiments and focus solely on
researching the behaviour of the nodes under high computational strain.

Another good idea is to use Linux Perf to gather the information about energy
consumption of the RAM of the servers. This would reduce the uncertainty of
the measurements done by Yokogawa WT310E on the entire nodes, giving more
informations about the behaviour of the servers power draw during tests,
therefore allowing to receive more precise results and form even better
conclusions.

Due to the fact, that many tests using different benchmarkshas has been
performed on various configurations, starting from the single thread benchmarks,
through the multi-CPUs and multi-GPUs and the finally hybrid benchmarks
utilizing all the resources available on the servers, another solution has been
proposed. Future goal is to create an empirical model, that will give insight
about how much will increase the difference between readings from the Yokogawa
external power meter and software solutions, such as Linux Perf and NVML\@.
The purpose of such model would be rapidly checking the link of the 
entire node power draw and the sum of power draws of active CPUs and GPUs.

Other suggestion is to give more attentions to the PSUs used in order to
power the servers. Many PSUs have different certifications, such as
\emph{Bronze}, \emph{Silver}, \emph{Gold}, \emph{Platinum} and \emph{Titanium}.
Their main difference is their energy efficiencies at various load levels.
The least efficient are the PSUs with \emph{Bronze} certification, while the
best are \emph{Titanium} ones. This could discover potential differencies
in the readings of power draw of the servers and could also benefit in
fine-tuning the previously mentioned empirical model.

As the culmination of this work, there is a high potential in writing
a research paper that would containg the most insightful informations
and conclusions made afer the tests. As a motivation for this idea it should
be noted, that this master thesis contains very exhausting experiments part,
with various implementations, benchmarks and class sizes tested as well as
meaningful conclusions drawn form the tests, than paired with previously
mentioned ideas for the expansion of the experiments, can form a strong base
for the research paper. The topic of energy aware computing is currently
very popular, due to increase of electricity use by the ICT sector world-wide,
and the rising demand of the \emph{green computing}, that has in mind finding
the best ratio of performance to energy use. Research paper based on this work
could contribute to this idea, due to verification of the software measurements
method upon which the power draw of the computational nodes is estimated.